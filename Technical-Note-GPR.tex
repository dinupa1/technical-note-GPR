% author = Dinupa Nawarathne
% email = dinupa3@gmail.com
% date = 10-01-2022

\documentclass{article}

% use packages
\usepackage[english]{babel}
\usepackage{amsmath}
% \usepackage{biblatex}


\addbibresource{Technical-Note-GPR.bib}


% title, author, date
\title{Gaussian Process Regression}
\author{Dinupa Nawarathne}
\date{October 01, 2022}

% document
\begin{document}

% title
\maketitle

% abstract
\begin{abstract}
Gaussian process regression (GPR) is a widely used learning techinique in machine learning.
\end{abstract}

% Introduction
\section{Introduction}

The Gaussian process model is a probabalistic supervised machine learning techinique used in classifiacation and regression tasks.

\section{Mathematical Baics}

Consider set of observerd data points. We want to fit a fucntion to represent these data points and then make a prediction at new data points. This is know as the regression. For a given set of obsereved data points, there are infinite number of possible functions that fit these data points. In GPR, Gaussian process conduct the regression by defining a distribution over these infinite number of functions.

\subsection{Gaussian Distribution}

A random variable $X$ is Gaussian or normally distributed with mean $\mu$ and variance $\sigma^{2}$ if its probability function(PDF) is \cite{kevin};

\begin{equation}
P_{X}(x) = \frac{1}{\sqrt{2\pi}\sigma}\exp{-\frac{(x-\mu)^{2}}{2\sigma^{2}}}
\end{equation}

where $X$ is the random variables and $x$ is the real argument.

% Phys. Rev. style
% \bibliographystyle{plain}
% \bibliography{Technical-Note-GPR}

\end{document}
